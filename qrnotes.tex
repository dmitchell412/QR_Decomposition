\documentclass{article}
\usepackage[left=1.in,right=1.in,top=1.in,bottom=1.in]{geometry}
\usepackage{listings}
\usepackage{color}
\geometry{letterpaper}

\definecolor{dkgreen}{rgb}{0,0.6,0}
\definecolor{gray}{rgb}{0.5,0.5,0.5}
\definecolor{mauve}{rgb}{0.58,0,0.82}

\lstset{frame=tb,
	language=C++,
	aboveskip=3mm,
	belowskip=3mm,
	showstringspaces=false,
	columns=flexible,
	basicstyle={\small\ttfamily},
	numbers=none,
	numberstyle=\tiny\color{gray},
	keywordstyle=\color{blue},
	commentstyle=\color{dkgreen},
	stringstyle=\color{mauve},
	breaklines=true,
	%breakwhitespace=true,
	tabsize=3
}

\begin{document}

\title{Accelerated and Validated Model Based MR Signal Reconstruction}
\author{Drew Mitchell\\
	Summer 2014 Tutorial Write-Up\\
	Faculty Member: R. Jason Stafford, Ph.D.}
\date{\today}
\maketitle

\begin{abstract}

Abstract. graphics processing unit (GPU)

\end{abstract}

\section*{Objectives}

The objective for this tutorial was to write kernels to perform linear algebra techniques, such as Gaussian elimination and QR decomposition, on the GPU in order to accelerate an MR signal reconstruction algorithm as it calls these subroutines. My personal objectives were to become proficient with C++, MATLAB, the CUDA parallel computing platform, and computational methods and algorithms commonly employed in MR signal reconstruction.

\section*{Background}

\subsection*{GPGPU Computing}
General purpose graphics processing unit (GPGPU) computation is an ideal way to accelerate reconstruction, because many of the subroutines of reconstruction involve parallelizable tasks which may be run simultaneously rather than serially. Traditional computation is performed with a central processing unit (CPU), which typically accomplishes one task per processor core in a linear manner. Most modern CPUs possess multiple cores, so some limited parallel processing is available. GPUs, on the other hand, contain many more cores capable of running hundreds of threads in parallel. In recent years, this parallel processing power has been harnessed for computation in many scientific fields because of its ability to accelerate common programming tasks by as much as a hundredfold.

For this application, a set of data is associated with each pixel of the image to be reconstructed. Calculations involving the data of one pixel may be performed independently of those for other pixels. Problems of this type are known as embarrassingly parallel. The computation is accelerated by performing calculations simultaneously on as many of the $256\times256$ pixels as possible. 

\subsection*{Reconstruction Algorithm}

\subsection*{Wavelets and Curvelets}

\section*{Methods/Techniques}

\subsection*{Root Solving by QR Decomposition}

To find the roots of a polynomial the Companion matrix method can be used. It can be verified by direct computation that the so called Companion matrix
$$\left(\begin{array}{ccccc}0 & 0 & \cdots & 0 & -c_0 \\1 & 0 & \cdots & 0 & -c_1 \\0 & 1 & \cdots & 0 & -c_2 \\\vdots & \vdots & \ddots & \vdots & \vdots \\0 & 0 & 0 & 1 & -c_{n-1}\end{array}\right)$$
has the characteristic polynomial $p(t)=c_0+c_1 t + \cdots + c_{n-1} t^{n-1} + t^n$. Thus the eigenvalues of the companion matrix are the roots of $p(t)$. To find the eigenvalues the QR algorithm can be used. The algorithm performs iterations of the form. 
$$A_{k+1} = R_{k} Q_{k},$$ 
where $Q_k$ is an orthogonal matrix and $R_k$ an upper triangular matrix, such that $A_k = Q_k R_k$, i.e. the QR decomposition of $A_k$, and $A_0=A$. It can be shown, that $A_k$ has the same eigenvalues as $A$ and that it converges to a triangular matrix, the {\it Schur form}.  Thus, the eigenvalues of $A$ and be read off the diagonal of $A_k$ after convergence.

\subsection*{Gram-Schmidt Orthogonalization}

One method for performing QR decomposition is Gram-Schmidt orthonormalization. This is a process for orthonormalizing a set of vectors in an inner product space, and the algorithm facilitates QR decomposition when applied to the column vectors of a matrix. If projection is abbrevated such that $\mathrm{proj}_u(v)=\frac{\langle u,v \rangle}{\langle u,u \rangle}u$, then the set of vectors $v$ are transformed to the set of orthogonal vectors $u$ by the following process:
$$\begin{array}{l}
u_1=v_1\\
u_2=v_2-\mathrm{proj}_{u_1}(v_2)\\
u_3=v_3-\mathrm{proj}_{u_1}(v_3)-\mathrm{proj}_{u_2}(v_3)\\
\vdots\\
u_k=v_k-\sum_{j=1}^{k-1}\mathrm{proj}_{u_j}(v_k)\\
\end{array}$$
The orthogonal vectors $u$ are normalized to the set of unit vectors $e_k=\frac{u_k}{\|u_k\|}$. When applying Gram-Schmidt orthonormalization to QR decomposition, the set of column vectors in $A$, such that $A=[a_1,\ldots,a_n]$, are orthogonalized.
$$\begin{array}{l}
u_1=a_1\\
u_2=a_2-\mathrm{proj}_{e_1}(a_2)\\
u_3=a_3-\mathrm{proj}_{e_1}(a_3)-\mathrm{proj}_{e_2}(a_3)\\
\vdots\\
u_k=a_k-\sum_{j=1}^{k-1}\mathrm{proj}_{e_j}(a_k)\\
\end{array}$$
Once the set of vectors $[a_1,\ldots,a_n]=A=QR$ have been orthonormalized, the orthogonal matrix $Q$ and upper triangular matrix $R$ are reconstructed as follows:
$$Q=[e_1,\ldots,e_n]$$
$$R=\left(\begin{array}{cccc}
\langle e_1,a_1 \rangle & \langle e_1,a_1 \rangle & \langle e_1,a_3 \rangle & \cdots \\
0 & \langle e_2,a_2 \rangle & \langle e_2,a_3 \rangle & \cdots \\
0 & 0 & \langle e_3,a_3 \rangle & \cdots \\
\vdots & \vdots & \vdots & \ddots
\end{array}\right)$$
The classical Gram-Schmidt process is numerically unstable. A modified Gram-Schmidt algorithm corrects this instability by also orthogonalizing $u_k^{(i)}$ against rounding errors in $u_k^{(i-1)}$:
$$\begin{array}{l}
u_k^{(1)}=v_k-\mathrm{proj}_{u_1}(v_k)\\
u_k^{(2)}=u_k^{(1)}-\mathrm{proj}_{u_2}(u_k^{(1)})\\
\vdots\\
u_k^{(k-2)}=u_k^{(k-3)}-\mathrm{proj}_{u_2}(u_k^{(k-3)})\\
u_k^{(k-1)}=u_k^{(k-2)}-\mathrm{proj}_{u_2}(u_k^{(k-2)})\\
\end{array}$$

The GPU kernels were written in C++ with calls to functions from the CUDA toolkit. The kernel itself was driven by a main program written in MATLAB for debugging before integration into the reconstruction algorithm. The efficiency of parallel structure of the kernel was analyzed using Nvidia Visual Profiler, an application included with the CUDA toolkit to aid in the efficient parallelization of code. The Gram-Schmidt orthogonalization and root solving kernel and the MATLAB driver program are provided in Appendix A.

\subsection*{The Francis Double-Shift Algorithm}

\subsection*{Reconstruction Using Wavelets and Curvelets}

\section*{Results}

The Gram-Schmidt orthogonalization kernel performs QR decomposition and has been parallelized successfully. The modified Gram-Schmidt algorithm reduces error and improves convergence as expected. However, certain matrices exist for which the Gram-Schmidt QR method does not converge, such as those which have roots that are complex conjugates of one another. 

\section*{Discussion}

The utilization of the CUDA platform has accelerated the root solving procedure as predicted. Unfortunately, no algorithm yet exists which can universally solve for the eigenvalues of any matrix. However, the physical constraints of the problem may be such that only well-behaved matrices need to be solved. The Francis double-step algorithm allows convergence even for complex conjugate root pairs, so the number of matrices which result in divergent behavior is smaller than with the Gram-Schmidt QR method. The Francis algorithm, along with several tricks to accelerate convergence, is the same one that MATLAB uses to perform the calculations for its \texttt{eig()} function. Continuing my work on this project, I will refine the parallelization of these algorithms to use GPU resources more efficiently and run more quickly, as well as parallelizing other subroutines to be run on the GPU. Ultimately, accelerated reconstruction algorithms like this make many MR signal reconstruction methods more feasible.

\begin{thebibliography}{9}

	\bibitem{cao14}
	Cao Z, Oh S, Otazo R, et al.
	Complex Difference Constrained Compressed Sensing Reconstruction for Accelerated PRF Thermometry with Application to MRI-Induced RF Heating.
	\emph{Magnetic Resonance in Medicine}.
	2014.

	\bibitem{farber11}
	Farber R.
	CUDA Application Design and Development.
	2011.

	\bibitem{fessler10}
	Fessler JA.
	Model-Based Image Reconstruction for MRI.
	\emph{IEEE Signal Process Mag.}
	2010;
	27(4):81-89.

	\bibitem{haack99}
	Haack EM.
	Magnetic Resonance Imaging: Physical Principles and Sequence Design.
	1999.

	\bibitem{hansen13}
	Hansen MS \& Sorensen TS.
	Gadgetron: An Open Source Framework for Medical Image Reconstruction.
	\emph{Magnetic Resonance in Medicine}.
	2013;
	69:1768-1776.

	\bibitem{munshi12}
	Munshi A.
	OpenCL Programming Guide.
	2012.

	\bibitem{wright14}
	Wright KL, Hamilton JI, Griswold MA, et al.
	Non-Cartesian Parallel Imaging Reconstruction.
	\emph{Journal of Magnetic Resonance Imaging}.
	2014.

\end{thebibliography}

\appendix
\section{Code}

\subsection{C++ QR Decomposition Kernel}

\begin{lstlisting}
#include <stdio.h>
#include <iostream>
#include <math.h>
#include <cusp/complex.h>
using namespace std;

typedef cusp::complex<double> cdouble;

__device__
void matrix_print(cdouble *mat, int nDim)
{
	for (int j = 0; j < nDim; ++j)
	{
		for (int i = 0; i < nDim; ++i)
		{
			printf(" %8.3f + %8.3fi", mat[j + i * nDim].real(), mat[j + i * nDim].imag());
		}
		printf("\n");
	}
	printf("\n");
}

__device__
cdouble dotprod(cdouble *vec1, cdouble *vec2, int nDim)
{
	cdouble x = 0;
	cdouble tmp = 0;
	for (int i = 0; i < nDim; ++i)
	{
		tmp.real(vec1[i].real());
		tmp.imag(-vec1[i].imag());
		x += tmp * vec2[i];
	}
	return x;
}

__device__
cdouble* matmult(cdouble *mat1, cdouble *mat2, int nDim)
{
	cdouble *x = new cdouble[nDim * nDim];
	for (int i = 0; i < nDim * nDim; ++i)
		x[i] = 0;
	for (int k = 0; k < nDim; ++k)
		for (int j = 0; j < nDim; ++j)
			for (int i = 0; i < nDim; ++i)
				x[j + k * nDim] += mat1[j + i * nDim] * mat2[i + k * nDim];
	return x;
}

__device__
cdouble l2norm(cdouble *vec, int nDim)
{
	cdouble x = 0;
	for (int i = 0; i < nDim; ++i)
		x += vec[i].real() * vec[i].real() + vec[i].imag() * vec[i].imag();
	x = sqrt(x);
	return x;
}

__device__
void transpose(cdouble *mat, int nDim)
{
	cdouble *x = new cdouble[nDim * nDim];
	for (int i = 0; i < nDim * nDim; ++i)
		x[i] = mat[i];
	for (int j = 0; j < nDim; ++j)
		for (int i = 0; i < nDim; ++i)
			mat[i + j * nDim] = x[j + i * nDim];
	delete[] x;
}

__device__
void make_comp_mat(cdouble *polynomial, cdouble *companion, int nDim)
{
	for (int i = 0; i < nDim * nDim; ++i)
		companion[i] = 0;
	for (int i = 0; i < nDim; ++i)
		companion[i * nDim] = -polynomial[i + 1] / polynomial[0];
	for (int i = 0; i < nDim - 1; ++i)
		companion[i * nDim + i + 1] = 1;
}

__device__
void select_diag(cdouble *vector, cdouble *matrix, int nDim)
{
	for (int i = 0; i < nDim; ++i)
		vector[i] = matrix[i * nDim + i];
}

__device__
void pixel_mat_select_1d(
	double *a_image_real,
	double *a_image_imag,
	cdouble *a,
	int nDim_matrix,
	int i_image)
{
	int offset_1d = i_image * nDim_matrix;
	for (int i = 0; i < nDim_matrix; ++i)
		a[i] = cdouble(a_image_real[i + offset_1d], a_image_imag[i + offset_1d]);
}

__device__
void pixel_mat_write_1d(
	double *a_image_real,
	double *a_image_imag,
	cdouble *a,
	int nDim_matrix,
	int i_image)
{
	int offset_1d = i_image * nDim_matrix;
	for (int i = 0; i < nDim_matrix; ++i)
	{
		a_image_real[i + offset_1d] = a[i].real();
		a_image_imag[i + offset_1d] = a[i].imag();
	}
}

__device__
void pixel_mat_select_2d(
	cdouble *a_image,
	cdouble *a,
	int nDim_matrix,
	int i_image)
{
	int offset_2d = i_image * nDim_matrix * nDim_matrix;
	for (int i = 0; i < nDim_matrix * nDim_matrix; ++i)
		a[i] = a_image[i + offset_2d];
}

__device__
void pixel_mat_write_2d(
	cdouble *Q_image,
	cdouble *R_image,
	cdouble *Q,
	cdouble *R,
	int nDim_matrix,
	int i_image)
{
	int offset_2d = i_image * nDim_matrix * nDim_matrix;
	for (int i = 0; i < nDim_matrix * nDim_matrix; ++i)
	{
		Q_image[i + offset_2d] = Q[i];
		R_image[i + offset_2d] = R[i];
	}
}

__device__
void gram_schmidt(cdouble *a, cdouble *Q, cdouble *R, int nDim)
{
	cdouble *u = new cdouble[nDim];
	cdouble *v = new cdouble[nDim];
	cdouble l2 = 0;
	for (int i = 0; i < nDim * nDim; ++i)
		Q[i] = R[i] = 0;
	for (int i = 0; i < nDim; ++i)
		u[i] = v[i] = 0;

	for (int k = 0; k < nDim; ++k)
	{
		for (int i = 0; i < nDim; ++i)
			u[i] = a[i + k * nDim];
		for (int j = k - 1; j >= 0; --j)
			for (int i = 0; i < nDim; ++i)
				u[i] -= R[j + k * nDim] * Q[i + j * nDim];
		l2 = l2norm(u, nDim);
		for (int i = 0; i < nDim; ++i)
			Q[i + k * nDim] = u[i] / l2;
		for (int j = k; j < nDim; ++j)
		{
			for (int i = 0; i < nDim; ++i)
			{
				u[i] = a[i + j * nDim];
				v[i] = Q[i + k * nDim];
			}
			R[k + j * nDim] = dotprod(u, v, nDim);
		}
	}

	delete[] u;
	delete[] v;
}

__device__
void modified_gram_schmidt(cdouble *a, cdouble *Q, cdouble *R, int nDim)
{
	cdouble *u = new cdouble[nDim];
	cdouble *v = new cdouble[nDim];
	cdouble prj = 0;
	cdouble l2 = 0;
	for (int i = 0; i < nDim * nDim; ++i)
		Q[i] = R[i] = 0;
	for (int i = 0; i < nDim; ++i)
		u[i] = v[i] = 0;

	for (int k = 0; k < nDim; ++k)
	{
		for (int i = 0; i < nDim; ++i)
			u[i] = a[i + k * nDim];
		for (int j = 0; j < k; ++j)
		{
			for (int i = 0; i < nDim; ++i)
				v[i] = Q[i + j * nDim];
			prj = dotprod(v, u, nDim);
			for (int i = 0; i < nDim; ++i)
				u[i] -= prj * v[i];
		}
		l2 = l2norm(u, nDim);
		for (int i = 0; i < nDim; ++i)
			Q[i + k * nDim] = u[i] / l2;
		for (int j = k; j < nDim; ++j)
		{
			for (int i = 0; i < nDim; ++i)
			{
				u[i] = a[i + j * nDim];
				v[i] = Q[i + k * nDim];
			}
			R[k + j * nDim] = dotprod(u, v, nDim);
		}
	}

	delete[] u;
	delete[] v;
}

__device__
void root_find(
	cdouble *polynomial,
	cdouble *root,
	int nDim_in,
	double tolerance,
	int upperbound)
{
	int nDim = nDim_in - 1;
	cdouble *a = new cdouble[nDim * nDim];
	cdouble *Q = new cdouble[nDim * nDim];
	cdouble *R = new cdouble[nDim * nDim];
	int nTol = 0;
	for (int i = 0; i < nDim; ++i)
		root[i] = 0;

	make_comp_mat(polynomial, a, nDim);

	for (int k = 0; k < upperbound; ++k)
	{
		modified_gram_schmidt(a, Q, R, nDim);
		a = matmult(R, Q, nDim);
		nTol = 0;
		for (int j = 0; j < nDim; ++j)
			for (int i = 0; i < nDim; ++i) {
				if (i > j && sqrt(a[i + j * nDim].real() * a[i + j * nDim].real() + 
					a[i + j * nDim].imag() * a[i + j * nDim].imag()) > tolerance) 
					++nTol; }
		if (nTol == 0) break;
	}

	select_diag(root, a, nDim);

	delete[] a;
	delete[] Q;
	delete[] R;
}

__global__
void QRDRoot(
	double *polynomial_image_real,
	double *polynomial_image_imag,
	double *root_image_real,
	double *root_image_imag,
	double const tolerance,
	int const upperbound,
	int const nDim_image,
	int const nDim_matrix)
{
	int i_image = blockDim.x * blockIdx.x + threadIdx.x;
	if (i_image > nDim_image * nDim_image) return;

	cdouble *polynomial = new cdouble[(nDim_matrix) * (nDim_matrix)];
	cdouble *root = new cdouble [(nDim_matrix - 1) * (nDim_matrix - 1)];

	pixel_mat_select_1d(polynomial_image_real, polynomial_image_imag, polynomial,
		nDim_matrix, i_image);
	root_find(polynomial, root, nDim_matrix, tolerance, upperbound);
	pixel_mat_write_1d(root_image_real, root_image_imag, root, nDim_matrix - 1, i_image);

	delete[] polynomial;
	delete[] root;
}
\end{lstlisting}

\subsection{MATLAB Driver}

\lstset{language=Matlab}
\begin{lstlisting}
%function driver

clear all
close all
format shortg

tol = 0.0001;
upbound = 1000;
nDim_image = 2;
nDim_matrix = 4;

h_root = complex(zeros(nDim_matrix-1,nDim_image,nDim_image),zeros(nDim_matrix-1,nDim_image,nDim_image));
h_poly = complex(randn(nDim_matrix,nDim_image,nDim_image),randn(nDim_matrix,nDim_image,nDim_image));
%h_a = randn(nDim_matrix,nDim_matrix,nDim_image,nDim_image);
%h_Q = zeros(nDim_matrix,nDim_matrix,nDim_image,nDim_image);
%h_R = zeros(nDim_matrix,nDim_matrix,nDim_image,nDim_image);

%for i = 1:nDim_image
%    for j = 1:nDim_image
%        h_a(:,:,i,j) = [1,1,0;1,0,1;0,1,1];
%    end
%end

%for i = 1:nDim_image
%    for j = 1:nDim_image
%        h_poly(:,i,j) = [1,-6,-72,-27];
%    end
%end

% transfer data to device
d_poly  = gpuArray( h_poly );
d_root  = gpuArray( h_root );
%d_a  = gpuArray( h_a );
%d_Q  = gpuArray( h_Q );
%d_R  = gpuArray( h_R );

qrdptx = parallel.gpu.CUDAKernel('qrd.ptx', 'qrd.cu');
threadsPerBlock = 256;
npixel = nDim_image*nDim_image;
qrdptx.ThreadBlockSize=[threadsPerBlock, 1, 1];
%blocksPerGrid = (npixel + threadsPerBlock -1) / threadsPerBlock;
%blocksPerGrid = (npixel  * threadsPerBlock - 1) / threadsPerBlock;
blocksPerGrid = ceil(npixel/threadsPerBlock);
qrdptx.GridSize=[blocksPerGrid, 1, 1];
%qrdptx.GridSize=[ceil(blocksPerGrid), 1, 1];
%qrdptx.GridSize=[256, 1, 1];

[dpolyrealout,dpolyimagout,drootrealout,drootimagout] = feval(qrdptx,real(d_poly),imag(d_poly),real(d_root),imag(d_root),tol,upbound,nDim_image,nDim_matrix);
%[daout,dQout,dRout] = feval(qrdptx,d_poly,d_Q,d_R,nDim_image,nDim_matrix);

hf_poly = gather(complex(dpolyrealout,dpolyimagout));
hf_root = gather(complex(drootrealout,drootimagout));

for i=1:nDim_image
	for j=1:nDim_image
		matsol(:,i,j)=roots(hf_poly(:,i,j));
		xcheck=norm(hf_root(:,i,j)-matsol(:,i,j))
	end
end

%for i=1:nDim_image
%    for j=1:nDim_image
%        xtest(:,i,j)=h_A(:,:,i,j)\h_b(:,i,j);
%        xcheck = norm(xtest(:,i,j)-dxout(:,i,j));
%        if xcheck >= 1.0e-8 
%            xcheck
%        end
%    end
%end

%exit
\end{lstlisting}

\end{document}
